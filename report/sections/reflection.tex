\section{Conclusion}

As written in previous parts of the the report, the P-Controller does not work standalone. It gets close to the operating point or even very close but with strong oscillation.

To counteract this Effect a PI-Controller is needed. This type of controller should be enough to control the system. The oscillation is rejected and the controller reaches it's set-point. The only disadvantage is the long rise time.

To counteract this effect once again, a PID-Controller is used. With the PID-Controller a slightly better rise time can be achieved. It comes with increased noise tho. Whilst the $T_d$ value makes the controller react faster, it also worsens the noise of the system.

To conclude it can be stated that a PI-Controller should be good enough for the purpose of controlling this system. Slightly better results can be achieved using a PID-Controller.
